\documentclass[oneside]{report}

\title{Linear Algebra Study Guide}
\author{Orgho A. Neogi}
\date{20 September 2017}

\usepackage{amsmath}
\makeatletter
\renewcommand*\env@matrix[1][*\c@MaxMatrixCols c]{%
  \hskip -\arraycolsep
  \let\@ifnextchar\new@ifnextchar
  \array{#1}}
\makeatother

\usepackage{graphicx}

\usepackage{float}

\usepackage[colorlinks=true]{hyperref}

\begin{document}

\maketitle

\section{Systems of Linear Equations}

A system of linear equations has

\begin{enumerate}
  \item no solution, or
  \item exactly one solution, or
  \item infinitely many solutions.
\end{enumerate}

A system of linear equations is said to be consistent if it has either one solution or
infinitely many solutions; a system is inconsistent if it has no solution.

Given an example System of Equations:

\begin{alignat}{8}
  &x_1 - &2x_2 + &x3  &= &0 \\
  &     &2x_2 - &8x_3 &= &8 \\
  &5x_1 &     - &5x_3 &= &10
\end{alignat}

The matrix of coefficients is:

\begin{center}
$\begin{bmatrix}
  &1 &2 &1\\
  &0 &2 &-8\\
  &5 &0 &-5
\end{bmatrix}$
\end{center}

The augmented matrix is:
\begin{center}
  $\begin{bmatrix}[ccc|c]
    1 &2 &1  &0\\
    0 &2 &-8 &8\\
    5 &0 &-5 &10
  \end{bmatrix}$
\end{center}

A system of linear equations can be solved using elementary row operations.

Elementary row operations are:
\begin{enumerate}
  \item (Replacement) Replace one row by the sum of itself and a multiple of another row.
  \item (Interchange) Interchange two rows.
  \item (Scaling) Multiply all entries in a row by a nonzero constant.
\end{enumerate}

Row operations can be applied to any matrix and are reversible.
\end{document}
